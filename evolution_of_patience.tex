\documentclass[titlepage, hidelinks, 12pt]{article}


%for custom page numbering:
\usepackage{fancyhdr}

\usepackage{lipsum}
\usepackage{hyperref}
\usepackage{palatino}
\usepackage{tikz}
\usepackage{chngcntr}
\counterwithin{figure}{section}
%\usepackage{breqn} % useful for breaking equations across multiple lines automatically. Breaks everything.


\usepackage{setspace}
%\usepackage{indentfirst} %tex default is no indent on first paragraph after heading
\usepackage{url}
\usepackage{amsmath, amssymb, amsfonts, amsthm}
\usepackage{float}
\usepackage{subfig}
\usepackage{graphicx}
\usepackage{environ, enumerate}
%\usepackage{mathbbol}
%\DeclareSymbolFontAlphabet{\amsmathbb}{AMSb}
\graphicspath{ {images/} }
\providecommand{\keywords}[1]{\textbf{\textit{Keywords---}} #1} 
\usepackage[format=plain,
            labelfont={bf, it},
            textfont=it]{caption}


\usepackage{epigraph}

% \epigraphsize{\small}% Default
\setlength\epigraphwidth{8cm}
\setlength\epigraphrule{0pt}

\usepackage{etoolbox}

\makeatletter
\patchcmd{\epigraph}{\@epitext{#1}}{\itshape\@epitext{#1}}{}{}
\makeatother



%%%%%%%%%
% indentation
%%%%%%%%%

\setlength\parindent{24pt}

\setlength{\voffset}{-1cm}
\setlength{\textwidth}{17cm}
\addtolength{\textheight}{2cm}
\setlength{\footskip}{1cm}
\addtolength{\oddsidemargin}{-2cm}
\addtolength{\evensidemargin}{-2cm}

\widowpenalty10000
\clubpenalty10000

%%%%%%%%%
% page numbering and logo
%%%%%%%%%

\pagestyle{fancy}

\fancyhead[L, C]{}
\fancyhead[R]{\includegraphics[scale=.35]{sb_logo_black.png}}
\fancyfoot[L]{}
\fancyfoot[C]{\thepage}
\fancyfoot[R]{}
\renewcommand{\headrulewidth}{0pt}
\renewcommand{\footrulewidth}{0pt}


%Def, Lemma, Theorem, Corollary environment
\theoremstyle{plain}
\newtheorem{theorem}{Theorem}[section]
\newtheorem{corollary}[theorem]{Corollary}
\newtheorem{lemma}[theorem]{Lemma}
\newtheorem{proposition}[theorem]{Proposition}
\newtheorem{conjecture}[theorem]{Conjecture}
\newtheorem{question}[theorem]{Question}
%\newtheorem*{proof}{Proof}
\theoremstyle{remark}
\newtheorem*{remark}{Remark}
\newtheorem*{example}{Example}
\theoremstyle{definition}
\newtheorem{definition}[theorem]{Definition}

%New commands
\newcommand{\Q}{\mathbb{Q}}
\newcommand{\Z}{\mathbb{Z}}
\newcommand{\N}{\mathbb{N}}
\newcommand{\R}{\mathbb{R}}
\newcommand{\T}{\mathcal{T}}
\newcommand{\E}{\mathbb{E}}
\newcommand{\betahat}{\hat{\beta}}

\newcommand{\varSS}{\frac{\partial \lambda}{ \partial b_{a}}}
\newcommand{\varLL}{\frac{\partial \lambda}{ \partial b_{a+1}}}
\newcommand{\LH}{\mathcal{LH}}




%New math operators
\DeclareMathOperator{\ringchar}{char}
\DeclareMathOperator{\diag}{diag}
\DeclareMathOperator*{\argmax}{argmax} \DeclareMathOperator{\disc}{disc}
\DeclareMathOperator{\MRS}{MRS}
\DeclareMathOperator{\smallersooner}{SS}
\DeclareMathOperator{\largerlater}{LL}
%\DeclareMathOperator{\exp}{exp}
\DeclareMathOperator{\hyp}{hyp}
\renewcommand\d[1]{\:\textrm{d}#1}
\newcommand*\diff{\mathop{\!\mathrm{d}}}

\doublespacing
\begin{document}
\begin{singlespace}
\begin{titlepage}
   \begin{center}
       \vspace*{1cm}
\Huge
       \textbf{The Adaptive Dynamics of Patience} \\
\Large
       \vspace{0.5cm}
      
            
       \vspace{4.1cm}
       \includegraphics[scale = 1.4]{sb_logo_black.png} \\
       \vspace*{\fill}
       \textbf{Eric, Christoph, and Jaye } 

            
       Department of Mathematics\\
       University of British Columbia
            
   \end{center}
\end{titlepage}





\begin{abstract}
    Time preferences describe an individual's willingness to wait for a delayed reward in the presence of a smaller/sooner alternative. 
    Much debate has been had between the validity of the exponential and hyperbolic discounting models of time preferences on theoretical 
    and empirical grounds. 
We put forward an evolutionary model which allows adaptive pressures to mould time preferences. 
We consider a trait substitution sequence, which assumes a population with uniform time preferences subject to small, random mutations which
either fixate or are eliminated. We use
standard results from adaptive dynamics to explicitly compute an evolutionarily stable set of time preferences to which our trait substitution sequence
converges. Our results imply that individual survival probability and ambient population growth determine risk and opportunity cost, respectively.
    \bigskip

    \begin{keywords}
        Time Preferences, Adaptive Dynamics, Exponential Discounting, Life History, Preference Reversals. 
    \end{keywords}

\end{abstract}

%\newpage

%\tableofcontents

\newpage

%\section*{Acknowledgements}
%\addcontentsline{toc}{section}{Acknowledgement}
%\newpage
%\input{texfiles/outline}
%\newpage
\end{singlespace}

\epigraph{Time is the most valuable thing a man can spend.}{---Theophrastus}
\section{Introduction}


Individuals regularly make decisions with delayed consequences. 
We prefer large rewards to smaller alternatives. At the same time, 
we are impatient, preferring rewards which come more immediately.
Of interest is when our desires for immediacy and quantity are in oposition, in the form of 
a smaller/sooner ($\smallersooner$) versus larger/later ($\largerlater$) dilemma. 
Common examples include retirement savings, choice of diet and 
exercise, and the management of addictions. All of these scenarios require one to make an immediate decision with delayed
ramifications. 


Time preferences may be compactly defined using discounting functions. 
A discounting function $\Delta(t)$ quantitatively describes the relationship between delay to reward and perceived value. A reward to be
received after delay $t$ maintains a fraction $\Delta(t)\in(0, 1)$ of its nominal value. We assume that immediatly received
rewards are not discounted at all, so that $\Delta(0) = 1$. Many studies
have considered why (in theory) and how (in practice) individuals evaluate time. Correspondingly, discounting functions
can be derived from mathematical models or from experimental data. 

Consider a simple model of time preferences.
If we assume that delayed rewards lose a constant fraction of their value for every unit-time delay, the result is the exponential
discounting model: 
\begin{equation}
    \Delta_{\exp}(t) = e^{-r t}.
\end{equation}
Here, $r>0$ is a parameter indicating how quickly a reward loses its value. The exponential discounting model is popular among economists
for its straightforward assumptions and ease of use \cite{sozou98}. 


Experimental results (for instance, see \cite{maia09, green04, doyle12}) find that observed discounting behaviour is better explained
by a hyperbolic, rather than exponential, discounting function:
\begin{equation}
    \Delta_{\hyp}(t) = \frac{1}{1+ht}.
\end{equation}
Experimentalists also find evidence of behaviour not permissible in the exponential discounting model. 
In \cite{read98} the authors find that
when deciding between a snack of chocolate and fruit, the delay before consumption has a significant effect on choice. Concretely, I prefer a chocolate bar
when I will consume immediately, but I prefer fruit salad if I am to snack after a week. This switch from impulsiveness to patience as
the consequences become further removed is called a preference reversal. 
Notably, exponential discounting is the only model of time preferences which does not predict preference reversals
\cite{strotz55}. 


\begin{figure}[H]
    \centering
    \includegraphics{discounting_appreciation.png}
    \caption{Appreciation is the reciprocal of discounting: an immediate reward of $P$ appreciates to value $F$ after delay $t$ whenever
        $P = \Delta(t)\cdot F$. 
        Hyperbolic and exponential time preferences are more easily distinguished by appreciation, rather than discounting, plots.
            Left: An exponential and a hyperbolic discounting function. Right: An appreciation plot. Exponential discounters exhibit exponential
        growth in appreciation, whereas hyperbolic discounting is consistent with linear growth.}
\end{figure}


%\begin{figure}[h]
%    \centering
%    \includegraphics[scale = 0.8]{present_value_pref_rev.png}
%    \includegraphics[scale = 0.8]{switching_point_pref_rev.png}
%    \caption{Preference reversals are inconsistent with models of exponential discounting, but are predicted by hyperbolic discounting. To illustrate,
%        consider both a hyperbolic discounter $\Delta_{\hyp}(t) = 1/(1+t)$ and exponential discounter $\Delta_{\exp}(t) = e^{-t}$ facing
%        dilemma $(t, F = 1.5)$, with time at dilemma $t$ variable. Top: the discounted value of smaller/sooner and larger/later at 
%        time at dilemma $t$. The hyperbolic discounter prefers the smaller/sooner option 
%        if and only if time at dilemma $t<1$. Bottom: each
%        point in $(t, F)$ space represents a dilemma, with the horizontal dotted line being all dilemmas $(t, F = 1.5)$. We plot
%        the patience threshold for a hyperbolic and exponential discounter; the smaller/sooner alternative is preferred if and only if
%        the black dotted line is below the patience threshold. Facing dilemma $(t, F=1.5)$, the hyperbolic discounter prefers the smaller/sooner
%        alternative if and only if $t < 1$. On the other hand, the exponential discounter's patience threshold is horizontal and 
%        never intersects the dotted line, resulting in no preference reversals.}
%        
%       \end{figure}


In practice, individuals rarely pick a discounting function or make decisions using mathematical models. 
Every decision we make is a consequence of millions of years of selective pressures and evolution. The adaptive unconscious \cite{wegner02} is 
the set of mental processes we use when making decisions without conscious and deliberate consideration. We start from the principle
that time preferences are determined by the adaptive unconscious,
which favours decisions aiding reproductive success. 
We may therefore conclude that individuals' instinctive time preferences are determined by an adaptive discounting function,
\begin{equation}
    \Delta^*(t).
    \label{eqn:adaptive_discounting}
\end{equation}

\begin{figure}[H]
    \centering
    \includegraphics[scale = 0.8]{discountedValueHyperbolic.png}
    \includegraphics[scale = 0.8]{discountedValueExponential.png}
    \caption{Preference reversals are predicted by hyperbolic discounting. Consider a hyperbolic ($h = 1$) and exponential ($r = 1$) discounter. 
       Suppose an agent
    decides between $\smallersooner$: [1 after delay $t$] and $\largerlater$: [1.5 at delay $t+1$]. The figure shows the discounted value
    of each alternative. The hyperbolic discounter prefers the smaller/sooner option if and only if the common delay $t < 1$. 
    Since the exponential discounter's curves never cross, they always prefer $\smallersooner$. }
\end{figure}

Our work focuses on determining adaptive time preferences. 
We follow in the footsteps of \cite{sozou98} and \cite{fawcett12}, which both reason along evolutionary lines. 
In the former, Sozou argues that individuals best adapted
to a constant, known risk will discount exponentially. He further shows that if the hazard rate is uncertain, non-exponential (and in particular,
hyperbolic) discounting may
be optimal. Sozou, however, neglects to consider opportunity cost, or the principle that time is inherently valuable. 
In \cite{fawcett12}, the authors explicitly account for both risk and opportunity cost, and derive several models of discounting predicting 
preference reversals. Their results, however, rely on very particular model setups, and each model produces a substantially different discounting
function. In brief, the technical details matter, and it is not obvious which setup is best.


%Indeed, \cite{gladwell05} has popularized the idea
%that careful deliberation does not always yield superior outcomes to snap decision making. 


\section{Life Histories and Reproductive Success}

The work in \cite{sozou98} and \cite{fawcett12} is limited by assuming that rewards translate directly to fitness outcomes. We instead
compute reproductive success
using standard fitness measures from evolutionary ecology: either lifetime progeny $R_0$ or monomorphic population growth rate $\lambda$ 
(see \cite{stearns92}).

Consider a population in which agents have the same survival and fecundity
schedules, $\vec{s} = (s_0, s_1, \ldots, s_{A-1}, s_A = 0)$, and $\vec{b} = (b_0, b_1, \ldots, b_A)$, where $A$ is the maximal age after
which death is guaranteed. The term $a$ indexes age, $s_a$ is the probability of surviving from age $a$ to $a+1$, and $b_a$ is the expected
number of offspring produced at age $a$. 
The survival/reproduction pair of schedules $\vec{s}, \vec{b}$
is called the population's life history, and is denoted $\LH_0$. 

Lifetime reproductive output is computed as
\begin{equation}
    R_0 = \sum\limits_{a = 0}^A \prod\limits_{k = 0}^{a-1} s_k b_a.
    \label{eqn:output}
\end{equation}
This is the sum of the expected number of offspring produced at each age $a= 0, 1, \ldots, A$. 
Population growth (also called the Lyapunov exponent) $\lambda$
appears in the discrete-time Euler-Lotka equation,
\begin{equation}
    \lambda^{A+1} - \sum\limits_{a = 0}^A \lambda^{A-a} \prod\limits_{k = 0}^{a-1}s_k b_k= 0.
    \label{eqn:euler-lotka}
\end{equation}


Whereas the expression for $R_0$ is easily computed, the Euler-Lotka equation
is a polynomial in $\lambda$ of degree $A+1$, and so has up to $A+1$ distinct solutions. The largest of those solutions is the Lyapunov exponent.
The Euler-Lotka equation is derived from the Leslie matrix describing population dynamics with life history $\LH_0$ (see \cite{stearns92}
for details). In practice, the Lyapunov exponent is computed numerically, and in each expression, the empty product evaluates to $1$. 
In the long run, a population with fixed life history will converge to a stable age distribution and
thus grow (or decay) 
exponentially at a constant rate $\lambda$ per unit time. If $\lambda = 1$, then the population remains at a constant size indefinitely.

We use population growth $\lambda$ as our measure of reproductive success. The assumption that populations grow exponentially is not without limitations.
By using $\lambda$, we implicitly assume that resources are in inexhaustible supply 
and do not limit growth. It therefore
may be useful to think of $\lambda \approx 1$ as the standard case, outside of which justification is required. Later, it is shown that assuming
$\lambda = 1$ is equivalent to using lifetime reproductive output $R_0$ to determine adaptive time preferences.
With these caveats taken care of, the question is now: What selective pressures does $\lambda$ 
exert over time preferences?


\section{Reproductive Dilemmas}
To model time preferences, we present each agent with a randomly determined
smaller/sooner versus larger/later fecundity dilemma.  As motivation, consider
a stochastically varying environment in which resource fluctuations result in the option to
increase reproduction at various points over the life history. Note that these fecundity increases are in addition to a baseline life history
$\LH_0$ common to all agents.  
Fix $\varepsilon > 0$ small.  
Individuals decide between:
\begin{enumerate}
    \item $\smallersooner:$ increase fecundity at age $a$ to $b_a\uparrow b_{a} + \varepsilon$, and
    \item $\largerlater:$ increase $b_{a+1}\uparrow b_{a+1} + F\cdot \varepsilon$.
\end{enumerate}
We use $\varepsilon > 0$ as our unit of fecundity dilemma, since we assume environmental stochasticity results in relatively small changes to 
fecundity. 
Denote each dilemma by the pair $(a, F)$, where $a$ is the age at which the dilemma is faced,
and $F$ is the ratio of $\largerlater$ to $\smallersooner$. 

\begin{figure}[H]
    \centering
    \includegraphics{evolutionary_dilemma.png}
    \caption{A baseline life history $\vec{s}, \vec{b}$ can be visualized as a bar plot. The bar at $a$ is the expected number of offspring at
        age $a$, given by
    $\prod_{k = 0}^{a-1} s_k b_a$. Evolutionary dilemmas are additions to fecundity. Illustrated is the dilemma $(2, 3)$, with
$\varepsilon = 0.01$. That is, an agent faces a choice between an additional $0.01$ offspring at age $2$ and an additional $0.03$ offspring
at age $3$.} 
\end{figure}

Individuals in our model have time preferences which indicate their preferred choice between $\smallersooner$ and $\largerlater$ facing any
dilemma. 
Consider a dilemma $(a, F)$. The switching point $x_a$ is the unique value satisfying that $\smallersooner$ is preferred
if and only if $F < x_a$. 
A patience threshold is a collection of switching points $(x_0, \ldots, x_{A-1})$.
Since a patience threshold indicates preferences facing any dilemma $(a, F)$, time preferences may be uniquely 
specified by either a patience threshold or a discounting function. We work with patience thresholds
for the mathematical analysis before translating back to discounting functions for further interpretation and discussion. 

\begin{figure}[H]
    \centering
    \includegraphics[scale = 0.8]{switching_point_pref_rev.png}
    \caption{Patience thresholds are mathematically convenient and provide a straightforward visual representation of time preferences. 
        Every point in $(a, F)$ space
    represents a dilemma at age $a$ with reward ratio $\largerlater : \smallersooner = F$. 
    Those points above the patience threshold correspond to dilemmas at which the $\largerlater$ alternative
is preferred. Those points below the threshold are when $\smallersooner$ is preferred. The patience threshold therefore also allows
us to visualize preference reversals. Consider the family of dilemmas $(a, 1.5)$, with age at dilemma $a$ variable (dotted black line). If the dilemma comes 
before age $a = 1$, the agent 
prefers $\smallersooner$. However, if the dilemma comes later in life, then $\largerlater$ is preferred. More generally, a
downward sloping patience threshold indicates preference reversals from $\smallersooner\to\largerlater$, and an upward sloping threshold
indicates the reverse. A horizontal patience threshold is consistent with exponential discounting, and indicates no preference reversals.
}
\end{figure}

By way of summary, our model includes a large (or infinite) population sharing the same baseline life history $\LH_0 = (\vec{s}, \vec{b}).$ 
Each agent faces a stochastically determined dilemma, denoted $(a, F)$, between a smaller/sooner and a larger/later fecundity reward. 
Time preferences are given by a patience threshold $(x_0, x_1, \ldots, x_{A-1})$. 
We proceed to consider the consequences of allowing an adaptive process to determine the patience threshold. 


\section{Adaptive Discounting}
Adaptive dynamics is a framework for mathematically describing the trajectory of a continuosly varying trait when subject to evolutionary pressures 
\cite{doebeli05}. In our case, we consider time preferences as encoded by each switching point $x_a$ of the patience threshold.
Adaptive dynamics makes two key assumptions. First, we assume that mutations are small. Second, mutations occur infrequently.
Under these meager assumptions, it can be shown that a mutant either goes extinct or replaces the resident population before any more mutations occur.
Therefore, a mutant with slightly different patience threshold is assumed to be invading into a population in which all agents share the same
time preferences. 
A value to which a trait converges is called convergent stable. A value to which a trait remains if already there is called evolutionarily stable.
Neither form of stability implies the other. For instance, a trait value which is convergent stable and not evolutionarily stable is called an
evolutionary branching point \cite{doebeli11}. We use adaptive dynamics
to derive a set of time preferences which we show to be both convergent stable and evolutionarily stable. 

Suppose a population has baseline life history $\LH_0$ and 
initial patience threshold $(x_0, x_1, \ldots, x_{A-1})$. Consider a mutant with almost the same time preferences, but
with switching point $y_a > x_a$ for some $a$. Then the mutant is slightly less patient than the resident.
The resident population and the mutant obtain identical outcomes except for dilemmas at age $a$ with ratio $F$ such that $x_a < F < y_a$, meaning
residents choose $\largerlater$ and the mutant chooses $\smallersooner$. 
For such a dilemma $(a, F)$, we analytically compute reproductive success via a first order Taylor
approximation: 
\begin{align}
    \lambda(\LH_0 + \smallersooner) &\approx \lambda(\LH_0) + \varepsilon \cdot \varSS \\
    \lambda(\LH_0 + \largerlater) &\approx \lambda(\LH_0) + F\varepsilon \cdot \varLL.
\end{align}
From the Taylor approximation we can identify the payoff from choosing each alternative, 
$\pi_{\smallersooner}$ and $\pi_{\largerlater}(F)$. Remark that the larger/later payout
varies as a function of $F$, whereas the smaller/sooner payout is constant. We have: 
\begin{align}
    \pi_{\smallersooner} &= \varepsilon\cdot \varSS \label{eqn:varss} \\ 
    \pi_{\largerlater}(F) &= F\varepsilon\cdot \varLL. \label{eqn:varll}
\end{align}

Invasion fitness describes the relative success of a rare mutant.  Mathematically, invasion fitness is the difference in payout received between a mutant 
and residents. 
Positive invasion fitness implies the mutant will fixate, whereas negative
invasion fitness implies the mutant will go extinct.
Since dilemmas
vary stochastically, invasion fitness is the \textit{average} difference in payout. Suppose the larger/later value $F$ is distributed with probability
density function $g(F)$. 
Denote by $\varphi_a(y_a, x_a)$ the invasion fitness of a mutant with switching point
$y_a$ into a resident population with switching point $x_a$.  
Recall that when $x_a < F < y_a$, residents choose the larger/later option and mutants prefer the smaller/sooner alternative. 
Invasion fitness is given by
\begin{align}
    \varphi_a(y_a, x_a) &= \mathbb{E}\left[ \pi_{\smallersooner} - \pi_{\largerlater}(F) \ | \ x_a < F < y_a\right] \\
    &= \int_{x_a}^{y_a} \left[ \pi_{\smallersooner} - \pi_{\largerlater}(F) \right]g(F) \d F. 
\end{align}
Remark that the expression for invasion fitness
$\varphi_a$ is the same whether $x_a < y_a$ or $y_a < x_a$, and is $0$ when $x_a = y_a$. 
Since dilemmas are distributed randomly, we compute the expected value as a weighted integral according to the distribution of $F$.

We can analyse invasion fitness $\varphi_a(y_a, x_a)$ mathematically to derive insights about adaptive traits. By differentiating, we compute
the selection gradient $D(x_a)$. Increasing the switching point $x_a$ is adaptive if and only if $D(x_a) > 0$. A trait value $x_a^*$ at which
$D(x_a^*) = 0$ is called singular, and $x_a^*$ may be convergent and/or evolutionarily stable. 
We compute:
\begin{align}
    D(x_a) &= \left.\frac{\partial }{\partial y_a} \right\vert_{y_a = x_a}\varphi_a(y_a, x_a)  \\
    &=\left.\frac{\partial }{\partial y_a} \right\vert_{y_a = x_a}  \int_{x_a}^{y_a} \left[ \pi_{\smallersooner} - \pi_{\largerlater}(F) \right]g(F) \d F\\
        &=\left[ \pi_{\smallersooner} - \pi_{\largerlater}(x_a)   \right]g(x_a) 
\end{align}
We verify that $D(x_a^*) = 0$ is solved whenever the payout from choosing smaller/sooner equals that from choosing larger/later,
$\pi_{\smallersooner} = \pi_{\largerlater}(x_a^*)$. Intuitively, this result is reasonable: the switching point which we will show to be evolutionarily
stable is the one at which the payout from smaller/sooner is the same as that from larger/later. 
We compute from equations \ref{eqn:varss} and \ref{eqn:varll} that
$x_a^* =  \varSS/\varLL$. 

We can further show that $x_a^*$ 
is both convergent stable and evolutionarily stable. Convergent stability follows from showing that $D'(x_a^*) < 0$, whereas evolutionary
stability follows from showing that $\frac{\partial^2 \varphi_a}{\partial y_a^2}(x_a^*, x_a^*)< 0$. In our case, the conditions are equivalent,
so we compute:
\begin{align}
    \frac{\partial^2 \varphi_a}{\partial y_a^2}(y_a, x_a) &=\frac{\partial}{\partial y_a}\left[\varSS \cdot \varepsilon  - \varLL \cdot y_a\varepsilon \right] g(y_a)\\
    &= \left[ -\varepsilon \varLL \right]\cdot g(y_a) + \left[ \varepsilon \varSS - y_a \right]
\end{align}
By evaluating at $(y_a, x_a) = (x_a^*, x_a^*)$, we determine the convergent and evolutionary stability of $x_a^*$. 
\begin{align}
    \frac{\partial^2 \varphi_a}{\partial y^2}(x_a^*, x_a^*) &= -\varepsilon \varLL g(x_a^*) + \left[ \varepsilon \varSS - \varepsilon \frac{\varSS}{\varLL}
    \varLL\right]g'(x_a^*) \\
    &= -\varepsilon \varLL g(x_a^*) + \left[0  \right]\cdot g'(x_a^*)\\
    &= -\varepsilon\varLL g(x_a^*) < 0
\end{align}
The final result is known to be negative. We have assumed $\varepsilon > 0$, and since $g$ is a probability density function, it is also
positive. Finally, since reproductive success $\lambda$ increases with fecundity, the derivative is also positive, and so the expression is
negative. 
Therefore, we conclude that the switching point will converge to $x_a^*$, and once there, it will not deviate. 

Remark that the switching point $x_a^* =  \varSS/\varLL$ has been computed specifically for dilemmas at age $a$. We therefore now extend to
the evolutionarily stable patience threshold $(x_0^*, x_1^*, \ldots, x_{A-1}^*)$. 
We may translate with ease between discounting functions and patience thresholds. Given a patience threshold $(x_0, x_1, \ldots)$, the corresponding
discounting function is given by 
\begin{equation}
    \Delta(t) = \left( \prod\limits_{k = 0}^{t - 1} x_k \right)^{-1}.
\end{equation}
Consequently, the evolutionarily stable discounting function (corresponding to our derived patience threshold) is
\begin{equation}
    \Delta^*(a) = \varSS/\frac{\partial \lambda}{\partial b_0}. 
    \label{eqn:stable_discounting}
\end{equation}
There is an important difference between the evolutionarily stable discounting function $\Delta^*(a)$ and a discounting function in the classical sense. 
The classical theory of time preferences sees discounting as age agnostic: delayed rewards are discounted to present value. The adaptive discounting
function $\Delta^*$, on the other hand, discounts delayed rewards to their value at age class $0$. The two views may be made equivalent by assuming
that all dilemmas are asked of our evolutionary agent when they are at age $0$. 




\section{Computation and Interpretation}
The Euler-Lotka equation can be leveraged to compute an expression for the evolutionarily stable switching points. By using
implicit differentiation, we compute
\begin{equation}
    \frac{\partial \lambda}{\partial b_a} = K\cdot \lambda^{-a} \prod\limits_{k = 0}^{a-1}s_k,
\end{equation}
where $K$ is an expression independent of $a$. We can now compute the evolutionarily stable switching point 
\begin{equation}
    x_a^* = \frac{\varSS}{\varLL} = \frac{K\cdot \lambda^{-a} \prod\limits_{k = 0}^{a-1}s_k}{K\cdot \lambda^{-a-1} \prod\limits_{k = 0}^{a}s_k} = \frac{\lambda}{s_a}.
\end{equation}
We further compute the adaptive discounting function
\begin{equation}
    \Delta^*(a) = \lambda^{-a}\prod\limits_{k = 0}^{a-1} s_k.
\end{equation}

Our approach lends itself to rich and organic interpretability. Previous aproaches to adaptive time preferences \cite{sozou98, fawcett12} explicitly
account for risk and/or opportunity cost as reasons to discount delayed rewards. It is therefore not surprising that their derived discounting
functions include terms corresponding to risk/opportunity cost. A pleasing feature of our life history approach to time preferences is that
we make no assumptions about why individuals should discount delayed rewards, but rather allow an evolutionary process to produce time preferences.  
Our results are nevertheless cleanly interpretable and point to both risk and opportunity cost as rationales for impatience.  

Remark that the adaptive dynamics approach in the previous section did not rely on the use of population growth $\lambda$ at all. We could therefore
run through the same argument to produce a discounting function resulting from lifetime reproductive output $R_0$ as our measure of success. In this
case, 
\begin{equation}
    \Delta_{R_0}^*(a) = \prod\limits_{k = 0}^{a-1} s_k.
\end{equation}
Note that this is the same expression as when computing with the Lyapunov exponent with $\lambda = 1$. From this observation we derive insight into
the reason for the presence of each term in the adaptive discounting function $\Delta^*$. 
\begin{enumerate}
    \item The product $\prod s_k$ accounts for \textit{risk}. Fecundity rewards later in life must be discounted since there is a risk that the
        agent not survive to benefit from them.
    \item The exponential term $\lambda^{-a}$ accounts for \textit{opportunity cost}. When time preferences evolve in the context of a growing
        population, there is adaptive pressure to prefer more immediate rewards. 
\end{enumerate}

We see that populations which are of constant size see no opportunity cost to delaying
reproduction, whereas growing populations face more pressure to reproduce earlier. 
From an ecological perspective, therefore, our theoretical work provides quantitative justification for the phenomenological demographic principle that
``a slowly growing population selects a later age of first reproduction than does a rapidly growing population,'' as 
described in \cite{bull04}.  

\begin{table}[h]
    \centering
\begin{tabular}{|l|l|}
    \hline
Population Size & Opportunity Cost            \\ \hline \hline
Constant             & None                            \\
Growing         & Reproduce Earlier \\
Shrinking       & Reproduce Later (Opportunity Benefit?) \\ \hline
\end{tabular}
\caption{When a population is growing ($\lambda > 1$), there is an opportunity cost to reproducing later in life. Of note is that 
when a population is dying out $(\lambda < 1)$, there is an opportunity benefit to delaying reproduction. If $\lambda < 1$, then in 
the discounting function
$\lambda^{-a}$ becomes larger as $a$ increases, so that an offspring produced late in life may be worth more than one produced earlier. If delayed
rewards are valued more than earlier rewards (meaning the discounting function is increasing rather than decreasing), we 
refer to such time preferences as inverted.}
\end{table}

\section{Examples and Preference Reversals}
Our results do not exclude the possibility of exponential discounting. Suppose that survival probability is constant, each individual
surviving from age $a$ to $a+1$ with probability $s\in(0, 1)$. Let $\lambda$ be the growth rate of the population. We therefore compute
that $\Delta^*(a) = (\lambda/s)^{-a}$. Just as in \cite{sozou98}, a constant, known hazard to survival results in exponential discounting
being adaptive. 

Our findings are also compatible with models of hyperbolic discounting, so long as the survival schedule is appropriately tuned. Suppose
that survival probability is given by $s_a = \frac{a+1}{a+2}$. If $\lambda$ is once more the Lyapunov exponent of the population, adaptive
discounting is given by the product $\Delta^*(a) = \lambda^{-a}\frac{1}{1+a}$, a combination of both exponential and hyperbolic discounting. 
If the population size is constant, then $\lambda = 1$ and time preferences are purely hyperbolic. More generally, if 
$h>0$ and $s_a = \frac{1+ha}{1 + h(a+1)}$, then adaptive discounting
is computed to be $\Delta^*(a) = \lambda^{-a}\frac{1}{1+ha}$. 

A major drawback of the exponential discounting model is that it precludes the possibility of preference reversals: the phenomenon that the
age at which a dilemma is faced may change preferences. While hyperbolic discounting allows for preference reversals, it is not unique in 
this respect. Recall that any
non-exponential model of time preferences predicts preference reversals. 
So long as survival probability is increasing in age (that is, $s_0 < s_1 < \ldots$), preference reversals are predicted from $\smallersooner\to
\largerlater$ as the age at dilemma increases. On the other hand, if survival probability decreases in age ($s_0 > s_1 > \ldots$), preference reversals are 
predicted from $\largerlater\to\smallersooner$. 

\begin{table}[h]
    \centering
\begin{tabular}{|l|l|}
    \hline
Survival Probability & Preference Reversals            \\ \hline \hline
Constant             & None                            \\
Increasing           & $\smallersooner\to\largerlater$ \\
Decreasing           & $\largerlater\to\smallersooner$ \\ \hline
\end{tabular}
\caption{Survival probability determines whether preference reversals occur, and in which direction. Intuitively, if survival 
decreases with age, then individuals are less willing to wait for delayed rewards later in life than they are early on. For instance,
facing a smaller/sooner versus larger/later dilemma at age $25$, I may be willing to wait for the larger/later reward. At the age of
$85$, however, the risk inherent in waiting means I prefer the smaller/sooner option.}
\end{table}







\begin{figure}[H]
    \centering
    \includegraphics{stacked_life_histories.png}
    \caption{Adaptive time preferences are determined (in part) by the baseline population growth rate $\lambda$. 
        We illustrate three life histories resulting in populations which are
        growing ($\lambda > 1$), stable ($\lambda = 1$), and decaying ($\lambda < 1$).}
\end{figure}
\newpage

\begin{figure}[H]
    \centering
    \includegraphics[scale = 0.8]{growth_dependence.png}
    \includegraphics[scale = 0.8]{switching_point_growth.png}
    \caption{We compute adaptive time preferences for three basline life histories. In all cases, survival probability is chosen to 
        produce hyperbolic discounting: $\Delta^*(a) = \lambda^{-a}\frac{1}{1+a}$. 
        Left: if a population grows at rate $\lambda = 6/5$ per unit time, appreciation looks exponential.
        Centre: If $\lambda = 1$, the population is stable in size, and appreciation looks linear, indicative of hyperbolic discounting. Right:
        If the population decays at rate $\lambda = 5/6$, inverted time preferences are adaptive, meaning delayed rewards may be valued more than
        more immediate alternatives. Bottom: The patience threshold is decreasing and the same shape for all values of $\lambda$. Preference reversals
    from smaller/sooner to larger/later are adaptive.}
\end{figure}
\newpage


\section{Application to Human Discounting}
In \cite{gurven07}, an evolutionary life history is put forward for homo sapiens. By studying modern hunter/gatherer societies, Gurven and Kaplan 
estimate survival and fecundity schedules for humans prior to the introduction of modern medicine, diet, and lifestyle. Specifically,
the authors approximate hazard by a Siler function, taking the form
\begin{equation}
    h_a = \alpha_1 e^{-\beta_1 a} + \alpha_2 + \alpha_3 e^{\beta_3 a}, 
    \label{eqn:siler}
\end{equation}
where $a$ is age measured in years. The corresponding survival function is then $s_a = 1 - h_a$. 

The Siler function accounts for elevated risk early in life, followed by a sustained
period of lower, constant risk, and finally steadily increasing risk late in life. By choosing the parameters $\alpha_j, \beta_k$ appropriately,
Gurven and Kaplan closely approximate the life histories of modern hunter/gatherer societies which we take as a proxy for human
evolutionary life history. From this life history, we derive time preferences from our adaptive discounting model. The life history 
derived from the Siler function gives time preferences which 
result in preference reversals early in life from $\smallersooner\to\largerlater$, and preference reversals later in life 
from $\largerlater\to\smallersooner$.
In between these two periods is a lengthy stretch during which discounting looks approximately exponential. 

\begin{figure}[h]
    \centering
    \includegraphics{evolved_preference_plot.png}
    \caption{Gurven and Kaplan \cite{gurven07} approximate homo sapiens evolutionary life history for modern hunter/gatherer societies. 
    We illustrate the evolutionarily stable time preferences resulting from the life history approximation for the Hadza society. Early in life,
preference reversals are predicted from smaller/sooner to larger/later. In mid-life, time preferences look approximately exponential. As death
draws nearer in the final third of the life history, patience diminishes and preference reversals are predicted from larger/later to smaller/sooner.}
\end{figure}





%\end{singlespace}


















\section{Conclusion}
We have herein presented a novel approach to computing discounting functions. 
By starting from the hypothesis that selective pressures determine time preferences,
we choose a measure of reproductive success and perform an adaptive dynamics analysis to conclude that evolutionarily stable time preferences are 
determined by life history factors and account
for both risk and opportunity cost. We exhibit that preference reversals, while often considered an irrationality, may in fact be an adaptive trait,
so long as risk of death is not constant in age. 

Our approach to adaptive dynamics does not make full use of the machinery available to us. Specifically, we assume that there is no competition between
individuals, so that the trait substitution sequence amounts to an optimization procedure. An exciting area for future study is to model the 
consequences of allowing an agent's likelihood of finding food to be determined not only by its own patience, but by the patience of
its competitors as well. 

\newpage
\bibliographystyle{alpha}
\bibliography{research}{}


\end{document}

